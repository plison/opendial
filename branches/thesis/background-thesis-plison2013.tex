
\chapter{Background}
\label{chap:background}

We introduce in this chapter the most importance concepts, methods and tools employed in the field of spoken dialogue systems -- with special emphasis on dialogue management.  To lay down the linguistic foundations of our work, we start by describing some of the key properties of dialogue.  A proper understanding of these aspects is indeed a prerequisite for the design of 
conversationally competent dialogue systems.

\section{Key properties of dialogue}

\note{reducing the communicative effort}

\footnote{An elliptical (also called non-sentential) utterance is a linguistic construction that lacks an overt predicate, such as \utt{where?}, \utt{at 8 o'clock}, \utt{a bit less, thanks} and \utt{brilliant!}. Their interpretation generally requires access to the dialogue history to recover their intended meaning \citep{Fernandez:2007}.}

\footnote{A deictic marker is a linguistic reference to an entity that is determined by the context of enunciation.  Examples of such markers are \utt{here} (spatial reference), \utt{yesterday} (temporal reference), \utt{this mug} (demonstrative) or \utt{you} (reference to a dialogue participant).}

\subsection{Turn-taking}

In spoken dialogue, the physical constraints of the communication medium impose that only one person speaks at a given time.


\note{mixed initiative}
\note{multi-party dialogue}

\note{ping-pong type of interactions Current systems rely on an overly simplistic model of the interaction, where each speaker take discrete turns with noticeable gaps in between, more resembling a walkie-talkie dialogue than a natural conversation. }

\note{Bohus paper}

\note{little silence in smooth dialogue:< 250ms Gaps less than actual sentence planning time - anticipate }

\note{At each TRP in each turn (Sacks 1974)
a
 
If speaker has selected A to speak, A must take floor
a
 
If speaker has selected no one to speak, anyone can
a
 
If no one else takes the turn, the speaker can
a
 
Selecting speaker A:
a
 
By explicit/implicit mention: What about it, Bob?
a
 
By gaze, function
a
 
Selecting others: questions, greetings, closing (notion of adjacency pairs)
a
 
(Traum et al., 2003) }

\subsection{Dialogue acts}

\note{called communicative acts, speech acts, dialogue move, etc.}

\note{Searle taxonomy, inspired by Austin}
mention non-sentential utterances or ellipsis
and references to context (deictics, anaphora, etc.)

\note{prosodic layer}

\subsection{Grounding}

\note{talk about feedbacks, common ground, and alignment}

\note{dialogue as collaborative activity}

\note{interpret each other's utterance cooperatively}


\section{Spoken Dialogue Systems}
 
\subsection{Architectures}

\note{beyond the boundaries of the isolated utterance}

\note{mention the question of adaptivity}

\subsection{Components}

\subsection{Applications}

\section{Dialogue Management}

\subsection{Hand-crafted approaches}

Some topics investigated in this paradigm are the semantic and pragmatic interpretation of dialogue moves \citep{ThomasonManuscript-THOEUA,Ginzburg2012}, the rhetorical structure of dialogue \citep{0521659515}, or the use of plan-based reasoning to infer the user intentions \citep{Allen1980,Litman87}.  These approaches are able to provide detailed analyses of various dialogue behaviours, but they generally assume complete observability of the dialogue context and provide only a very limited account (if any) of uncertainties.


\note{Finite-state automata, frame-based, logical reasoning, etc.}

\subsection{Statistical approaches}


This is typically done by representing the dialogue domain as a Markov Decision Process (MDP) or Partially Observable Markov Decision Process (POMDP) and subsequently estimating the parameters of these models from data \citep{Supelec270}. 

 related to partial observability (noisy spoken inputs, unknown user intentions) and stochastic action effects (the user behaviour can be hard to predict). 
 
  Probabilistic modelling techniques must however face two important challenges. The most pressing issue is the paucity of appropriate data sets.  Statistical models often require large amounts of training data to estimate their parameters. Unfortunately, real interaction data is scarce, expensive to acquire, and difficult to transfer from one domain to another.  Moreover, many domains exhibit a rich internal structure with multiple tasks to perform, sophisticated user models, and a complex, dynamic context.  In such settings, the dialogue system might need to track a large number of variables in the course of the interaction, which quickly leads to a combinatorial explosion of the state space.  
  
\note{MDP and POMDPs}

\section{Summary}
