\chapter*{Acknowledgements}

The present thesis is the product of five years of work. Many of the ideas behind this dissertation were first sketched during my stay at the DFKI Language Technology Lab in Saarbr\"ucken (from 2008 to 2010), and then further developed upon starting a PhD at the Department of Informatics of the University of Oslo (from 2011 until now).  Needless to say, this project could not have been completed without the help and support of the many people that accompanied me on this long research journey.

My first thanks go to my thesis supervisors: Stephan Oepen, Erik Velldal and Geert-Jan M. Kruijff.  Their guidance and encouragements on my research work have been invaluable.  They also contributed with numerous advices and suggestions on multiple revisions of this dissertation and played a key role in helping me improve the form and content of the final draft. I am particularly grateful to Stephan and Erik for welcoming me to the Language Technology Group (LTG) and for accepting to supervise a thesis topic which ventured outside the group's traditional areas of expertise. I am equally indebted to Geert-Jan, who was my research supervisor while working at DFKI and kindly accepted to serve as co-supervisor of this thesis when I relocated to Oslo and enrolled in the PhD program there. Geert-Jan  also taught me a lot about how to conduct research, write papers, give talks, and participate in large collaborative projects.

I naturally want to thank my colleagues for the great working environment and friendly atmosphere during all these years.  I especially wish to express my gratitude to all my LTG workmates for making me feel welcome and part of this fantastic group from day one. Their specialisation in multiple areas of natural language processing also helped me broaden my research interests and pique my curiosity for other fields of investigation.  Likewise, I want to thank my former colleagues at DFKI for their friendliness and for many inspiring discussions on dialogue systems and human--robot interaction that contributed to shaping the direction of this thesis. My colleagues from the EU-funded projects CogX and ALIZ-E also deserve a special mention for enlightening me on many aspects of software engineering and cognitive robotics.

The thesis greatly benefited from exchanges with other international researchers, and in particular with the regular participants to the ``Young Researchers' Roundtable on Spoken Dialogue Systems'' (YRRSDS) organised every year to foster discussions and collaborations between young researchers working in this exciting field. I enjoyed many stimulating discussions with Timo Baumann, Nina Dethlefs, Jana G\"otze,  Matt Henderson, Casey Kennington, Alejandra Lorenzo, Raveesh Meena, Lina Rojas, Ethan Selfridge and William Yang Wang on various topics related to dialogue processing.  I would especially like to thank Timo, Casey, Matt, Raveesh and Nina for accepting to read through parts of my thesis draft. I am similarly indebted to Amanda Stent and Heriberto Cuay\'{a}huitl for their highly valuable feedback on several of my conference papers.

As any researcher in robotics will tell you, working with real robots can be at times a frustrating experience and brings a host of low-level technical challenges, from broken motors and defective sensors to countless software integration issues. I was fortunate to conduct my experiments with the Nao robot from Aldebaran Robotics.  I am grateful to the Department of Informatics for allowing us to acquire this robot and to Aldebaran Robotics for making available this excellent robotic platform and for always going the extra mile when it comes to technical support. I also wish to acknowledge the support of the NOTUR high-performance computing facilities employed for our simulation studies. Finally, I want to thank all the participants in the Wizard-of-Oz interactions and user trials, as none of the experiments presented in this thesis would have been remotely possible without their active contribution.

Last but certainly not least, I am without words to thank my wife and life partner, Caroline, for all her love and support through the years, and for being the reason I moved to Norway in the first place. Thank you for everything.

\vspace{1cm}

\begin{flushright}Pierre Lison \\ Oslo, 30th October, 2013. \end{flushright}

\null 
\vfill
\begin{footnotesize}\subsubsection*{Typesetting of this thesis}
The thesis is written with \LaTeXe \ based on a document layout prepared by the University of Oslo. The bibliographical references are generated with \textsc{Bib}\negthinspace\TeX.  The keyword index at the end of this dissertation are produced with the help of \begin{scriptsize}\textsf{makeindex}\end{scriptsize}. The diagrams and data plots are drawn using Apple's \textit{Pages} and \textit{Numbers} productivity tools, while the mathematical figures are rendered with the \textit{MATLAB} numerical computing environment.\end{footnotesize}
