\chapter*{Acknowledgements}


\note{INCOMPLETE}

%The present thesis is the final product of five years of research. Many of the ideas behind this dissertation were first sketched during my work at the DFKI Language Technology Lab in Saarbr\"ucken (from 2009 to 2010), and then further developed after enrolling in a PhD at the Department of Informatics of the University of Oslo (from 2011 to 2013).  Needless to say, this project could not have been completed without the help and support of the great many people that accompanied me on this long research journey.

%My first thanks go to my thesis supervisors: Stephan Oepen, Erik Velldal and Geert-Jan M. Kruijff.  Their constant support and encouragements on my research work have been invaluable.  They also contributed with numerous advices and suggestions on multiple revisions of this dissertation and considerably helped improving the form and content of the final draft. I am particularly grateful to Stephan and Erik for welcoming me to the Language Technology Group (LTG) and for accepting to supervise a thesis topic which ventured outside the group's traditional areas of expertise. I also owe a debt to Geert-Jan, who was my research supervisor while working at DFKI and kindly accepted to serve as co-supervisor of this thesis when I relocated to Oslo and enrolled in the PhD program there. Geert-Jan also taught me a lot about how to conduct high-quality research, write papers, give talks, and participate in large collaborative projects.

%I naturally want to thank my great colleagues for the working environment and friendly atmosphere during all these years.  I especially want to express my gratitude to all my LTG colleagues for making me feel welcome and part of  this fantastic group from day one. Their expertise in multiple areas of natural language processing also helped me broaden my research interests and pique my curiosity for other fascinating fields of investigation.  My former workmates at DFKI (and in particular Mira Jani\v{c}ek, Ivana Kruijff-Korbayov\'{a}, Raveesh Meena, Sergio Roa, and Hendrik Zender) also deserve a special mention for their friendliness, and for many inspiring discussions on dialogue modelling and human--robot interaction that contributed to shaping the direction of this thesis. My colleagues in the EU-funded projects CogX and ALIZ-E also deserve credit for teaching me the few things I know about cognitive robotics and decision-theoretic planning.

%The thesis also benefited from multiple exchanges with other researchers, and in particular with the participants of the ``Young Researchers' Roundtable on Spoken Dialogue Systems'' (YRRSDS) organised every year to foster discussions and collaborations between young researchers working in this exciting field. I enjoyed the stimulating discussions with Timo Baumann, Casey Kennington, Matthew Henderson, and Nina Dethlefs on topics such as incremental parsing, statistical modelling and reinforcement learning. They also provided insightful comments on my thesis draft. I also wish to explicitly express my gratitude to Heriberto Cuay\'{a}huitl for his highly valuable feedback on several of my papers.

%As any researcher in robotics will tell you, working with robots can be at times a frustrating experience and brings a host of low-level technical challenges (from broken motors and defective sensors all the way to software-related issues).  I was fortunate to conduct my experiments with the Nao robot from Aldebaran Robotics.  Aldebaran Robotics deserves credit for developing this great robotic platform and for going the extra mile when it comes to technical support. I am also grateful to the Department of Informatics for allowing us to purchase this robot in the first place and to the NOTUR high-performance computing infrastructure which was instrumental to the simulation experiments in Chapter \ref{chap:rllearning}. Finally, I wish to thank all the participants in the Wizard-of-Oz and user evaluation studies -- none of the interaction experiments would have been possible without their active contribution.





%wish to thank

% guidance?

% be fortunate to

% instrumental

%support, encouragement, insights, feedback, invaluable

%want to express my warmest thanks

%support

% praise

% stimulating discussions

% deserve credit

%benefited from

%feedback that considerably improved the quality of this dissertation.

% Caro for the love and support

%\note{Thank in particular YRRSDS gang (Raveesh, Timo, Matthew, Casey), Mira, Heriberto,}

%\note{mention the HPC infrastructure, Aldebaran robotics, the participants to the experiment}

\vspace{1cm}

\begin{flushright}
Pierre Lison \\
Oslo, October 30th 2013.
\end{flushright}

\null
\vfill
\begin{description}
\item [Typesetting: ] The thesis is written with \LaTeXe \ based on a document layout prepared by the University of Oslo. The bibliographical references are generated with \textsc{Bib}\negthinspace\TeX and the keyword index with \textsf{makeindex}. The diagrams and data plots are drawn using Apple's \textit{Pages} and \textit{Numbers} applications, while the mathematical figures are rendered with the \textit{MATLAB} numerical computing environment.
\end{description}