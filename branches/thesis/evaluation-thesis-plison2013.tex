\chapter{User evaluation}
\label{chap:user-evaluation}

\note{ANOVA, since we would have two baselines and our approach? (see e.g. Passonneau's article)}

\section{Wizard-of-Oz data collection}

\note{the domain and how the data was collected}

\section{Learning setup}

\note{three competing models}

\section{User trials}

\note{various measures (number of turns, duration, number of disconfirm)}

\note{direct assessment of user satisfaction}


% note about our approach: generalisation enable a better account of the data sparsity problem.  plus, the state dynamics are not lost since we perform belief update.  Finally, the appraoch can be seen as an initial boostrapping that can then be further refined through online reinforcement learning (Bayesian prior), as in Williams etc. also, we learn utilities, not a direct classification. Also: a user simulator is difficult for situated and open-ended environments.  we learn a POMDP policy by simulation
