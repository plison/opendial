\chapter*{Abstract}

This thesis presents a new modelling framework for dialogue management based on the concept of \textit{probabilistic rules}.  Probabilistic rules are defined as \textit{if...then...else} constructions associating logical conditions on input variables to probabilistic effects on output variables.  These rules function as high-level templates for the generation of a directed graphical model. Their expressive power allows them to represent the probability and utility models employed in dialogue management in a compact and efficient manner. As a consequence, they can drastically reduce the amount of interaction data required for parameter estimation as well as enhance the system ability to generalise over unseen situations. Furthermore, probabilistic rules can also be used to encode domain-specific constraints and assumptions into statistical models of dialogue, thereby enabling system designers to incorporate their expert knowledge of the problem structure.  Due to their principled combination  of logical and probabilistic reasoning, we argue that probabilistic rules are particularly well suited to devise hybrid models of dialogue management that can account for both the complexity and uncertainty that characterise many dialogue domains.

%In order to update the dialogue state and perform action selection, probabilistic rules are instantiated at runtime as latent nodes of a directed graphical model.   

The thesis also demonstrates how the parameters of probabilistic rules can be efficiently estimated using both supervised and reinforcement learning techniques. In the supervised learning case, the rule parameters are learned by imitation on the basis of small amounts of Wizard-of-Oz data.  Alternatively, rule parameters can also be optimised via trial and error from repeated interactions with a (real or simulated) 
user. Both learning strategies rely on Bayesian inference to iteratively estimate the parameter values and provide the best fit for the observed interaction data. Three consecutive experiments conducted in a human--robot interaction domain attest to the practical viability of the proposed framework and its advantages over traditional approaches.  In particular, the empirical results of a user evaluation with 37 participants show that a dialogue manager structured with probabilistic rules outperforms both purely hand-crafted and purely statistical methods on an extensive range of subjective and objective metrics of dialogue quality.

The modelling framework presented in this thesis is implemented in a new software toolkit called \opendial{}, which is made available to the research community and can be used to develop various types of spoken dialogue systems based on probabilistic rules. 
