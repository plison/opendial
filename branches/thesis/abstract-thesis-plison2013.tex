\chapter*{Abstract}

This thesis presents a new modelling framework for dialogue management based on the concept of \textit{probabilistic rules}.  Probabilistic rules are defined as \textit{if...then...else} constructions which associate logical conditions on input variables to probabilistic effects on output variables. The expressive power of these rules allows them to encode probability and utility models in a highly compact manner and thus drastically reduce the amount of interaction data required for parameter estimation. They can also be employed to incorporate domain-specific constraints and assumptions into probabilistic models of dialogue, thereby enabling system designers to leverage their expert knowledge of the problem structure.  In order to update the dialogue state and perform action selection, probabilistic rules are instantiated at runtime as latent nodes of a directed graphical model.  They can hence be viewed as representing high-level templates for a classical probabilistic model. Due to its principled combination of logical and probabilistic reasoning, we argue that the formalism of probabilistic rules is particularly well suited to devise hybrid models of dialogue management that can account for both the complexity and uncertainty that characterise most dialogue domains.

The thesis also demonstrates how the parameters of these probabilistic rules can be efficiently estimated using both supervised and reinforcement learning techniques. In the supervised learning case, the parameters are learned by imitation on the basis of small amounts of Wizard-of-Oz data.  Alternatively, rule parameters can also be optimised via trial-and-error from repeated interactions with a (real or simulated) user. Both learning strategies rely on Bayesian inference to iteratively estimate the parameter values to provide the best fit for the observed interaction data. Three consecutive experiments conducted in a human--robot interaction domain attest to the practical viability of the proposed framework and its advantages over traditional approaches.  The empirical results of an extensive user evaluation with 37 participants show in particular that a dialogue manager based on probabilistic rules outperforms both purely handcrafted and statistical methods on a broad range of subjective and objective metrics of dialogue quality.

The modelling framework presented in this thesis is implemented in a new software toolkit called \opendial{}, which is made available to the research community and can be used to develop various types of spoken dialogue systems based on probabilistic rules. 
