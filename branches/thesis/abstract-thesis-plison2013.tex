\chapter*{Abstract}

This thesis presents a new modelling framework for dialogue management based on the concept of \textit{probabilistic rules}.  Probabilistic rules are formally defined as \textit{if...then...else} constructs that map logical conditions on input variables to probabilistic effects on output variables. The expressive power of these rules allows them to represent probability and utility models in a highly compact manner and therefore drastically reduce the amount of interaction data required for parameter estimation. In addition, the framework also enables system designers to directly incorporate domain-specific constraints and assumptions into probabilistic models of dialogue.  In order to update the dialogue state and perform action selection, probabilistic rules are instantiated at runtime as latent nodes of a directed graphical model.  They can thus be viewed as providing high-level templates for a classical probabilistic model. Thanks to its combination of logical and probabilistic inference, the formalism of probabilistic rules is well suited to devise hybrid models of dialogue management that account for both the complexity and uncertainty that characterise most dialogue domains.

The thesis demonstrates how the parameters of these probabilistic rules can be efficiently estimated using both supervised and reinforcement learning techniques. In the supervised learning case, the parameters are learned by imitation on the basis of small amounts of Wizard-of-Oz data.  Alternatively, rule parameters can also be optimised via trial-and-error from repeated interactions with a (real or simulated) user. Both learning strategies rely on Bayesian inference to iteratively estimate the parameter values that result in the best fit for the observed interaction data. Three consecutive experiments conducted in a human--robot interaction domain attest to the practical viability of the proposed framework and its advantages over traditional approaches.  An extensive user evaluation with 37 participants shows in particular that a dialogue manager based on probabilistic rules outperforms both purely handcrafted and statistical methods on a broad range of subjective and objective metrics of dialogue quality.

